\chapter{Results}\label{chapter:Results}
Now that the models are set and run, the RMSE are collected it is time to evaluate the results of the different models. The results will be evaluated in two ways. The first is the RMSE of the training set. The lower the error is, the better the model fits the test. But using only this method is not a good indicator whether the model will fit future data. This is why in a second evaluation the difference between the error of the training set is compared to the test set. The test set, including the training data, is run through the algorithm the same way as the training set to calculate the error.\newline
The evaluation is done for all orders as well as for a restaurant. Since yum2take was chosen to be the best example with the best data, the evaluation of the results is done on this restaurant.\newline
For each model, the best results are presented in a table. Since there are many options for the weighted moving average and simple exponential smoothing, only the lowest error and thus best result from the training set is displayed. This is done since when the forecast would be done, only the present result can be used and thus the lowest error would be chosen. The table contains the algorithm used in the most left column and two columns for each order set, for yum2take on the left and for all orders on the right. For each set there are the names of each set and the number of orders it contains as well as the difference between the error of the two sets.
\section{No time categorization}\label{section:No time categorization}
This forecast was done by using all previous orders for the current forecast. The results are shown in Tabular \ref{tab:No time categorization results}.
\begin{table}[h]
\centering
\caption{No time categorization results}
\label{tab:No time categorization results}
\scalebox{0.5}{%
\begin{tabular}{lllllll}
\hline
              & \multicolumn{3}{l}{yum2take}                                                                                                 & \multicolumn{3}{l}{all}                                                                                                        \\ \hline
Algorithm     & \begin{tabular}[c]{@{}l@{}}training\\ \#686\end{tabular} & \begin{tabular}[c]{@{}l@{}}test\\ \#945\end{tabular} & difference & \begin{tabular}[c]{@{}l@{}}training\\ \#2129\end{tabular} & \begin{tabular}[c]{@{}l@{}}test\\ \#3196\end{tabular} & difference \\ \hline
MA            & 6.02755                                                  & 5.950609                                             & -0.07694   & 8.409876                                                  & 8.565466                                              & 0.15559    \\ \hline
MA (650/2125) & 5.738142                                                 & 5.880966                                             & 0.142823   & 4.542798                                                  & 8.976687                                              & 4.433889   \\ \hline
SES (0.1/0.1) & 6.116109                                                 & 5.973192                                             & -0.14291   & \multicolumn{1}{l}{8.560794}                             & \multicolumn{1}{l}{8.65649}                          & 0.095695   \\ \hline
\end{tabular}}
\end{table}
\newline\textbf{All orders}\newline
\newline\newline\textbf{Only yum2take orders}\newline
The forecasts results in a RMSE of 6 minutes for the moving average model. The weighted moving average is calculated in different weights. The weights start with 25 orders and increase step by step to 675 orders. The RMSE of these forecasts is stable between 6 minutes and 6.5 minutes and the best result is achieved with a weight of 675 orders. The best RMSE for the simple exponential smoothing is the result of an $\alpha$ of 0.1. When increasing $\alpha$ the RMSE also increases to up to 8.5 minutes of RMSE. The accuracy improves for all models when using the test set by 0.07 minutes for the moving average, between 0.07 minutes up to 0.45 minutes when using the weighted version and between 0.14 minutes and 0.42 minutes for the simple exponential smoothing.
\section{No time but slot categorization}\label{section:No time but slot categorization}
After separating the orders into their slots chronological, only previous orders from this slot were used for the forecast. The results are shown in Tabular \ref{tab:No time but slot categorization results}.
\begin{table}[h]
\centering
\caption{No time but slot categorization results}
\label{tab:No time but slot categorization results}
\scalebox{0.5}{%
\begin{tabular}{lllllll}
\hline
             & \multicolumn{3}{l}{yum2take}                                                                                                 & \multicolumn{3}{l}{all}                                                                                                        \\ \hline
Algorithm    & \begin{tabular}[c]{@{}l@{}}training\\ \#686\end{tabular} & \begin{tabular}[c]{@{}l@{}}test\\ \#945\end{tabular} & difference & \begin{tabular}[c]{@{}l@{}}training\\ \#2129\end{tabular} & \begin{tabular}[c]{@{}l@{}}test\\ \#3196\end{tabular} & difference \\ \hline
MA           & 5.980634                                                 & 5.943105                                             & -0.03752   & 8.203962                                                  & 8.514927                                              & 0.310964   \\ \hline
MA (25/100)  & 6.165134                                                 & 6.017991                                             & -0.14714   & 7.957781                                                  & 8.492864                                              & 0.535082   \\ \hline
SES (0.1/0.1) & 6.065397                                                 & 5.982731                                             & -0.08266   & 8.379257                                                  & 8.618189                                              & 8.618189   \\ \hline
\end{tabular}}
\end{table}
\newline\newline\textbf{All orders}\newline
The results for this models are around 8 minutes which is still too high but better than without the use of slots.
 The difference of RMSE for each set is again with around half a minute neglectable.
\newline\newline\textbf{Only yum2take orders}\newline
The results for yum2take the moving average results in a RMSE of little under 6 minutes. The weighted average is used with 25 and 50 order steps from which the 25 orders scores better with 6.2 minutes while 50 orders has 6.3 minutes as RMSE. The simple exponential smoothing also has a good results with close to 6 minutes for an $\alpha$ of 0.1. When increasing $\alpha$ the RMSE grows to almost 8 minutes. The accuracy between training and test set is almost zero for the moving average algorithm, improves by 0.15 minutes for the weighted moving average and also improves for all simple exponential smoothing models with a range from 0.08 minutes up to 0.47 minutes while decreasing the accuracy.
\section{Day categorization}\label{section:Day categorization}
The results in Tabular \ref{tab:Day categorization} are generated for categorization by day where only previous orders of the current day are taken into account for the forecast of the next value. There is no weighted moving average model since there has not been enough data to do forecast in a convenient manner.
\begin{table}[h]
\centering
\caption{Day categorization results}
\label{tab:Day categorization}
\scalebox{0.5}{%
\begin{tabular}{lllllll}
\hline
             & \multicolumn{3}{l}{yum2take}                                                                                                 & \multicolumn{3}{l}{all}                                                                                                        \\ \hline
Algorithm    & \begin{tabular}[c]{@{}l@{}}training\\ \#686\end{tabular} & \begin{tabular}[c]{@{}l@{}}test\\ \#945\end{tabular} & difference & \begin{tabular}[c]{@{}l@{}}training\\ \#2129\end{tabular} & \begin{tabular}[c]{@{}l@{}}test\\ \#3196\end{tabular} & difference \\ \hline
MA           & 6.011955                                                 & 5.993232                                             & -0.01872   & 8.403266                                                  & 8.57489                                               & 0.171623   \\ \hline
SES (0.1/0.1) & 6.080652                                                 & 5.995665                                             & -0.08498   & 8.434747                                                  & 8.57399                                               & 0.139242           \\ \hline
\end{tabular}}
\end{table}
\newline\newline\textbf{All orders}\newline
When forecasting on daily basis most results are about 8.5 minutes. This is not an improvement over the categorizations before, even though the error decreased.
\newline\newline\textbf{Only yum2take orders}\newline
When using day and slot categorization the moving average performes well again, compared to the other models, with an RMSE of 6 minutes. The result improves for the test set by 0.08 minutes. Simple exponential smoothing returned 6.1 minutes for an $\alpha$ of 0.1 and only slowly increased to 6.9 minutes for the maximum of $\alpha$ 1. The accuracy also improved for up to 0.4 minutes from the training to the test set.
\section{Day and slot categorization}\label{section:Day and slot categorization}
For this the data set was split into the three slots for each day. The results are shown in Tabular \ref{tab:Day and slot categorization results}.
\begin{table}[h]
\centering
\caption{Day and slot categorization}
\label{tab:Day and slot categorization results}
\scalebox{0.5}{%
\begin{tabular}{lllllll}
\hline
             & \multicolumn{3}{l}{yum2take}                                                                                                 & \multicolumn{3}{l}{all}                                                                                                        \\ \hline
Algorithm    & \begin{tabular}[c]{@{}l@{}}training\\ \#686\end{tabular} & \begin{tabular}[c]{@{}l@{}}test\\ \#945\end{tabular} & difference & \begin{tabular}[c]{@{}l@{}}training\\ \#2129\end{tabular} & \begin{tabular}[c]{@{}l@{}}test\\ \#3196\end{tabular} & difference \\ \hline
MA           & 6.011955                                                 & 5.993232                                             & -0.01872   & 8.162479                                                  & 8.535424                                              & 0.372944   \\ \hline
MA (15/60)   & 6.127325                                                 & 6.109218                                             & -0.0181    & 7.859824                                                  & 8.737695                                              & 0.87787    \\ \hline
SES (0.1/0.1) & 6.04016                                                  & 6.104262                                             & 0.064101   & 8.220405                                                  & 8.581225                                              & 0.360819   \\ \hline
\end{tabular}}
\end{table}
\newline\newline\textbf{All orders}\newline
This categorization is an improvement over the one without slot categorization. This shows that going for a finer categorization can help to improve results. But since the results are still too high with 8 minutes and for the test set they even increase, this is not a usable categorization.
\newline\newline\textbf{Only yum2take orders}\newline
For yum2take the results are better than for the forecast which uses all orders. The moving average has the best result with about 6 minutes. When weighted it increases to 6.1 minutes, for both weights, namely 15 and 30 orders, which is not much difference to the training set result. Same goes for the simple exponential smoothing, where the RMSE and the difference between the sets grows the bigger $\alpha$ gets. The difference between training set is for all models neglectable because it is around 0.5 minutes, which is too small to have an impact on the real life scenario.
\section{Week categorization}\label{section:Week categorization}
The results of categorization by weekday is shown in Tabular \ref{tab:Week categorization result}.
\begin{table}[h]
\centering
\caption{Week categorization result}
\label{tab:Week categorization result}
\scalebox{0.5}{%
\begin{tabular}{lllllll}
\hline
              & \multicolumn{3}{l}{yum2take}                                                                                                 & \multicolumn{3}{l}{all}                                                                                                        \\ \hline
Algorithm     & \begin{tabular}[c]{@{}l@{}}training\\ \#686\end{tabular} & \begin{tabular}[c]{@{}l@{}}test\\ \#945\end{tabular} & difference & \begin{tabular}[c]{@{}l@{}}training\\ \#2129\end{tabular} & \begin{tabular}[c]{@{}l@{}}test\\ \#3196\end{tabular} & difference \\ \hline
MA            & 6.074067                                                 & 5.967819                                             & -0.10624   & 8.402317                                                  & 8.568181                                              & 0.165863   \\ \hline
MA (8/16)     & 6.120197                                                 & 5.949482                                             & -0.17071   & 8.164492                                                  & 8.814434                                              &            \\ \hline
SES (0.2/0.2) & 6.008943                                                 & 5.981932                                             & -0.02701   & 8.399327                                                  & 8.580201                                              & 0.180873   \\ \hline
\end{tabular}}
\end{table}
\newline\newline\textbf{All orders}\newline
When separating by week the resulting RMSE is around 8.4 minutes which is almost the level of the other models and not usable. Also the difference increases for each algorithm, but only by a neglectable amount.
\newline\newline\textbf{Only yum2take orders}\newline
As before, the results for yum2take are better with a time frame from around 6 minutes instead of 8 minutes which is due to the more specific data. The moving average has one of the best results with around 6.1 minute. The moving average ranges from 6.1 to 6.6 minutes. The worst results come from the smallest (4 orders, 6.4 minutes) and greatest (16 orders, 6.6 minutes) while the weight of 8 and 12 orders has an RMSE of around 6.1 minutes. When using the simple exponential smoothing, the results range from 6.0 minutes for an $\alpha$ of 0.2 and slowly grows up to 6.29 for an $\alpha$ of 1. The difference between training and test data set negative for most cases, meaning the result improves for the test set.
\section{Week and slot categorization}\label{section:Week and slot categorization}
In Tabular \ref{tab:Week and slot categorization results} the results for week and slot categorization are shown.
\begin{table}[h]
\centering
\caption{Week and slot categorization results}
\label{tab:Week and slot categorization results}
\scalebox{0.5}{%
\begin{tabular}{lllllll}
\hline
              & \multicolumn{3}{l}{yum2take}                                                                                                 & \multicolumn{3}{l}{all}                                                                                                        \\ \hline
Algorithm     & \begin{tabular}[c]{@{}l@{}}training\\ \#686\end{tabular} & \begin{tabular}[c]{@{}l@{}}test\\ \#945\end{tabular} & difference & \begin{tabular}[c]{@{}l@{}}training\\ \#2129\end{tabular} & \begin{tabular}[c]{@{}l@{}}test\\ \#3196\end{tabular} & difference \\ \hline
MA            & 6.048462                                                 & 6.013698                                             & -0.03476   & 8.21474                                                   & 8.532601                                              & 0.31786    \\ \hline
MA (8/12)     & 6.137753                                                 & 5.952349                                             & -0.1854    & 8.133761                                                  & 8.843515                                              & 0.709754   \\ \hline
SES (0.3/0.3) & 5.935083                                                 & 6.155115                                             & 0.220032   & 8.18856                                                   & 8.615477                                              &  0.426916          \\ \hline
\end{tabular}}
\end{table}
\newline\newline\textbf{All orders}\newline
When forecasting on weekday and slot basis the RMSE there is still not much difference to the other models. The values are still between 8.2 minutes and 8.7 minutes which is not acceptable for a forecast like this.
\newline\newline\textbf{Only yum2take orders}\newline
The forecast for the models including the categorization of week and slot returned 6.0 minutes for the moving average. The weighted moving average can be calculated for steps of 4, 8 and 12 weeks as weight. It returns 6.3 minutes, 6.1 minutes and 6.2 minutes for the weights. These four results improve their accuracy when forecasting with the test set by between 0.03 and 0.13 minutes. For the simple exponential smoothing an $\alpha$ between 0.2 and 0.5 gives good results of little under 6 minutes and growing for $\alpha$ over 0.5 to a RMSE of 6.45 minutes. The accuracy decreases for all values by about 0.1 to 0.2 minutes except for an $\alpha$ of 1 where it improves by 0.08 minutes.
\section{Weekday categorization}\label{section:Weekday categorization}
Weekday categorization results in the following values shown in Tabular \ref{Weekday categorization results}.
\begin{table}[h]
\centering
\caption{Weekday categorization results}
\label{Weekday categorization results}
\scalebox{0.5}{%
\begin{tabular}{lllllll}
\hline
              & \multicolumn{3}{l}{yum2take}                                                                                                 & \multicolumn{3}{l}{all}                                                                                                        \\ \hline
Algorithm     & \begin{tabular}[c]{@{}l@{}}training\\ \#686\end{tabular} & \begin{tabular}[c]{@{}l@{}}test\\ \#945\end{tabular} & difference & \begin{tabular}[c]{@{}l@{}}training\\ \#2129\end{tabular} & \begin{tabular}[c]{@{}l@{}}test\\ \#3196\end{tabular} & difference \\ \hline
MA            & 6.208329                                                 & 6.07587                                              & -0.13245   & 8.582432                                                  & 8.660576                                              & 0.078144   \\ \hline
MA (8/12)     & 6.181677                                                 & 6.105938                                             & -0.07573   & 8.08649                                                   & 8.700854                                              & 0.614363   \\ \hline
SES (0.2/0.3) & 6.23239                                                  & 6.113957                                             & -0.11843   & 8.573267                                                  & 8.683558                                              &  0.110291          \\ \hline
\end{tabular}}
\end{table}
\newline\newline\textbf{All orders}\newline
The results of weekday categorization are also in the same value spectrum then the other models, ranging from around 8 minutes up to 8.7 minutes as RMSE.
\newline\newline\textbf{Only yum2take orders}\newline
Weekday categorization returns 6.2 minutes for the moving average model. It improves by 0.13 minutes for the test set. The weighted moving average has three different weights, 4, 8 and 12 weekdays in a row. The results are 6.7 minutes, 6.2 minutes and 6.5 minutes and they improve by 0.4 minutes for the first and thirds and by 0.07 for the second when used on the test set. When using simple exponential smoothing for this categorization the results start with 6.3 minutes of RMSE for all $\alpha$ under 0.6 and grow up to 6.9 minutes when $\alpha$ is 1. The accuracy improvement from training to test set is around 0.1 minute and then growns equaly to 0.36 minutes.
\section{Weekday and slot categorization}\label{section:Weekday and slot categorization}
\begin{table}[h]
\centering
\caption{Weekday and slot categorization results}
\label{Weekday and slot categorization results}
\scalebox{0.5}{%
\begin{tabular}{lllllll}
\hline
              & \multicolumn{3}{l}{yum2take}                                                                                                 & \multicolumn{3}{l}{all}                                                                                                        \\ \hline
Algorithm     & \begin{tabular}[c]{@{}l@{}}training\\ \#686\end{tabular} & \begin{tabular}[c]{@{}l@{}}test\\ \#945\end{tabular} & difference & \begin{tabular}[c]{@{}l@{}}training\\ \#2129\end{tabular} & \begin{tabular}[c]{@{}l@{}}test\\ \#3196\end{tabular} & difference \\ \hline
MA            & 6.063212                                                 & 6.13718                                              & 0.073968   & 8.448001                                                  & 8.73583                                               & 0.287828   \\ \hline
MA (12/12)    & 4.202483                                                 & 5.802201                                             & 1.599717   & 8.117922                                                  & 8.913511                                              & 0.795589   \\ \hline
SES (0.3/0.2) & 6.319075                                                 & 6.163683                                             & -0.15539   & 8.378466                                                  & 8.720113                                              &            \\ \hline
\end{tabular}}
\end{table}
\newline\newline\textbf{All orders}\newline
\newline\newline\textbf{Only yum2take orders}\newline
The weekday and slot categorization has an RMSE of 6.1 minutes for the moving average and 6.8 minutes, 6.3 minutes and an outlier of 4.4 minutes for the weighted moving average with an weight of 4, 8 and 12 weeks as weight. The difference between training set and test set is 0.07 minutes of decrease for the moving average and 0.36 minutes of improve for the first two weights of the weighted moving average. The outlier decreases its accuracy by 1.6 minutes to 5.8 minutes which is still a good result. The results for the simple exponential smoothing are around 6.4 minutes for $\alpha$ from 0.1 to 0.6 and growing to 6.77 minutes for bigger $\alpha$. The difference is between plus and minus 0.1 minute for these models.
\section{Single slot categorization}\label{section:Single slot categorization}
For this categorization the minislot was introduced, dividing the day into 30 minutes slots which are forecasted seperately. The results are shown in Tabular \ref{Single slot categorization results}. There is no weighted moving average since some slots do not have enough data to calculate this model.
\begin{table}[h]
\centering
\caption{Single slot categorization results}
\label{Single slot categorization results}
\scalebox{0.5}{%
\begin{tabular}{lllllll}
\hline
              & \multicolumn{3}{l}{yum2take}                                                                                                 & \multicolumn{3}{l}{all}                                                                                                        \\ \hline
Algorithm     & \begin{tabular}[c]{@{}l@{}}training\\ \#686\end{tabular} & \begin{tabular}[c]{@{}l@{}}test\\ \#945\end{tabular} & difference & \begin{tabular}[c]{@{}l@{}}training\\ \#2129\end{tabular} & \begin{tabular}[c]{@{}l@{}}test\\ \#3196\end{tabular} & difference \\ \hline
MA            & 6.206459                                                 & 6.078068                                             & -0.12839   & 8.122773                                                  & 8.408003                                              & 0.28523    \\ \hline
SES (0.1/0.2) & 6.347103                                                 & 6.149942                                             & -0.19716   & 8.670448                                                  & 8.752016                                              &  0.081567          \\ \hline
\end{tabular}}
\end{table}
\newline\newline\textbf{All orders}\newline
This categorization results in a relatively low moving average RMSE but the other result are still high.
\newline\newline\textbf{Only yum2take orders}\newline
The last simple categorization is by slot of the day. The moving average model returns 6.2 minutes RMSE and an improvement in accuracy of 0.12 minutes when using the test test set.
\section{Combination of Categorizations}\label{section:Combination of Categorizations}
The results for the combined model are treated differently. Since there are over thirty thousand combinations for the weights and algorithm used, only the best results were looked at. It was decided to extract the 5 best results for the test set. The results are saved into an excel file including the algorithm used and the weighting of the different factors.
\newline\newline\textbf{All orders}\newline
When searching for the lowest error on the training set, the result is about 8.947 minutes. It is the result of different combinations. The most occurring one is a mix from 20\% of the forecast of the current slot and 25\% of the forecast from the current day. The rest is mostly created from the last 7 day’s forecast as well as the slot in the last 4 weeks. These combinations produce the most accurate forecasts. Overall there are only good forecasts generated by the normal moving average. When using the test set to compare the result to the training set the accuracy drops by around 0.06 minutes for the top five models.
\begin{table}[h]
\centering
\caption{Combination of Categorizations results for all orders}
\label{Combination of Categorizations results for all orders}
\scalebox{0.5}{%
\begin{tabular}{llllllllll}
\hline
yum2take  & \multicolumn{3}{l}{Results}                 & \multicolumn{6}{l}{Weights}                                                                           \\ \hline
Algorithm & training      & test          & difference  & \multicolumn{3}{l}{slot} & \multicolumn{3}{l}{all orders}                                             \\ \hline
          & \#2129 orders & \#3196 orders &             & today & 7 days & 4 weeks & today & 7 days & \begin{tabular}[c]{@{}l@{}}weekday\\ 4 weeks\end{tabular} \\ \hline
MA        & 8.947472944   & 9.009768419   & 0.062295475 & 20    & 10     & 15      & 25    & 30     & 0                                                         \\ \hline
MA        & 8.947526044   & 9.009875609   & 0.062349565 & 20    & 5      & 20      & 25    & 30     & 0                                                         \\ \hline
MA        & 8.947812087   & 9.00973973    & 0.061927643 & 20    & 15     & 10      & 25    & 30     & 0                                                         \\ \hline
MA        & 8.947971379   & 9.010061298   & 0.062089919 & 20    & 0      & 25      & 25    & 30     & 0                                                         \\ \hline
MA        & 8.948138981   & 9.009792703   & 0.061653722 & 20    & 10     & 15      & 25    & 25     & 5                                                         \\ \hline
\end{tabular}}
\end{table}
\newline\newline\textbf{Only yum2take orders}\newline
The results show that the best combination for yum2take consists of 25\% current slot’s forecast, 15\% of the current day\’s forecast and 20\% of the forecast of the slot for the predecessing 7 days. The rest of weight is combined from the other forecasts and all forecasts are using moving average as algorithm. This results in an error of about 6.26 minutes. The improvement of the test set over the training set is around 0.15 minutes for all of the five best results.
\begin{table}[h]
\centering
\caption{Combination of Categorizations results for yum2take}
\label{Combination of Categorizations results for yum2take}
\scalebox{0.5}{%
\begin{tabular}{llllllllll}
\hline
all orders  & \multicolumn{3}{l}{Results}                & \multicolumn{6}{l}{Weights}                                                                           \\ \hline
Algorithm & training     & test         & difference   & \multicolumn{3}{l}{slot} & \multicolumn{3}{l}{all orders}                                             \\ \hline
          & \#686 orders & \#945 orders &              & today & 7 days & 4 weeks & today & 7 days & \begin{tabular}[c]{@{}l@{}}weekday\\ 4 weeks\end{tabular} \\ \hline
MA        & 6.265421304  & 6.118386791  & -0.147034513 & 25    & 20     & 15      & 15    & 0      & 25                                                        \\ \hline
MA        & 6.265551527  & 6.118048866  & -0.147502661 & 25    & 20     & 15      & 15    & 5      & 20                                                        \\ \hline
MA        & 6.265856228  & 6.117902793  & -0.147953435 & 25    & 15     & 20      & 15    & 0      & 25                                                        \\ \hline
MA        & 6.265868384  & 6.117507337  & -0.148361047 & 25    & 15     & 20      & 15    & 5      & 20                                                        \\ \hline
MA        & 6.265946422  & 6.117772134  & -0.148174288 & 25    & 20     & 15      & 15    & 10     & 15                                                        \\ \hline
\end{tabular}}
\end{table}

\section{All orders versus yum2take}\label{section:All orders versus yum2take}
It can be observed that the results for yum2take restaurant are always lower, except in some outliers cases, than when using all orders. For the restaurant the results of the forecast models are always around 6 minutes while when using all orders the resulting RMSE is about 8 minutes. The reasons and how to deal with it or use it has to be discussed in the conclusion.
