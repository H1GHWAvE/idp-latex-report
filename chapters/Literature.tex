\chapter{Review of Literature and Research}\label{chapter:Review of Literature and Research}
\section{Research}\label{section:Research}
In order to get a good overview over the problem, a search in Google Scholar was done. The goal was to find similar problems and approaches to it. Different words were chosen for the search, like "preparation time", "meal", "restaurant" and "forecast". The results did not give the right output for this problem. Most of the forecasting in restaurant was about the amount of staff needed, the amount of meals which will be sold over a period or the size a buffet has to
be. These topics did not fit the problem given because the problem stated is pretty unique. On the one hand, most of the delivery services have their own fleet of drivers and are not that time critical as volo is. On the other hand has not yet a company combined a food delivery fleet with a routing algorithm. This is why a custom approach had to be made.\newline
A plan with different steps was created. As a first step, a solid base about forecasting had to be made by searching for basics of forecasting. In the second step, this knowledge should be applied to the case before evaluating it in a third step.
\section{Approach and Basics}\label{section:Approach and Basics}
After searching for fitting materials, the book "Forecasting: Principles and Practice" by R. Hyndman and A. Athanasopoulos was chosen as a starting point, since it covers the topic pretty good for forecasting beginners. After reading the book it was determined to follow the steps the book suggests to create the forecast. The book had five basic steps for forecasting.\newline
First of all, you have to define the problem. This is simple but often the most time consuming part of the whole task. The forecaster has to talk to everybody who uses the forecast as well as the people who support the data to generate it. Questions regarding the way it will be used, by who it is required and how it fits into the process of the ones who required it have to be asked. Also the people who support the forecast with collecting data, maintaining the data and use the forecast for future planning have to be figured out. After the problem is defined, in a second step, all information available is gathered. This requires to get all statistical data available, but often there will be not enough historical data to create a decent statistical model. The available data is combined with the expertise of the people who collected this data and use the forecasts. In the next step, preliminary analysis is done. The data is put into a graph to get a first rough overview over the behavior. This is a simple way to see patterns, trends, seasonalities or business cycles. Outliers can be spotted and can be questioned early. Now the real forecasting can begin. In the next step the model is chosen and fitted to the data. The models are created from the historical data and all other knowledge which can support the forecast. Different potential models are used and later compared. This is done in the last step, when the model is used and evaluated. This is done by comparing the outcomes of the different models with the actual events which are forecasted.\newline
Now that there is a basic approach, more information about the different models is needed. This information is taken from PowerPoint slides which are provided by the Department of Logistics. The slides are from lectures with forecasting as a topic and are used to get a rough overview over different simple algorithms often used for forecasting.

\subsection{Time Series}\label{subsection:Time Series}
Since the events are happening on a time line, time series decomposition was the choice. Time series can have different patterns that is why putting the data in a graph first is very useful. The book and the slides likewise addressed common patterns and methods which can be used on time series and improve forecasts.
The time series can then be split into different parts to support the corresponding pattern. There are three types of time series patterns. The first pattern is the trend. The trend is an decrease or increase over a long period of time. The second second is the season pattern. It is influenced by seasonal factors and has a known and fixed time period and. The third one is the cyclic pattern. Cyclic pattern influence usually has fluctuations of at least 2 years and the period is not fixed. The available data should be transformed to a time series and divided into suitable components, in case patterns are present.
\subsection{Algorithms for Forecasting}\label{subsection:Algorithms for Forecasting}
Now that the time series is investigated for patterns, it is time to forecast. The book and slides offer different algorithms to fulfill this task. Three algorithms are chosen to be presented more closely.
\subsubsection{Moving Average}\label{subsubsection:Moving Average}
A classical method is the moving average. It is used to calculate the next forecast value of a time series iteratively. For the time series it is done by taking all prior events in account for the next value. The calculated values are independent. The formula for the average at time t is: m(t)= 1/n E (n-1, 0) x(t-1)
\subsubsection{Weighted Moving Average}\label{subsubsection:Weighted Moving Average}
The window which is taken into the calculation of the next value can also be weighted. In contrast to the normal Moving Average, a window for the amount of values included in the creation of the average value. The window moves with the time series. When a calculating the next value, the first value is removed from the window. This way only a specific amount of values stays in the calculation.
\subsubsection{Simple Exponential Smoothing}\label{subsubsection:Simple Exponential Smoothing}

Cite papers with \verb+\cite{BibTeX key}+\\
A comprehensive introduction to Operations Research: \cite{Winston.2007}
