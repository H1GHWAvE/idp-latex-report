\chapter{Review of Literature and Research}\label{chapter:Review of Literature and Research}
This chapter is about the sources and the information gained before starting the actual forecasting.
\section{Research}\label{section:Research}
In order to get a good overview over the problem, a search in Google Scholar was done. The goal was to find similar problems and approaches to it. Different words were chosen for the search, like "preparation time", "meal", "restaurant" and "forecast". The results did not give the right output for this problem. Most of the forecasting in restaurant was about the amount of staff needed, the amount of meals which will be sold over a period or the size a buffet has to
be. These topics did not fit the problem given because the problem stated is pretty unique. There are not many companies who do last mile delivery with time critical materials. Very often the restaurant has its own fleet of driver which is available all the time and they send it when a meal has to be delivered. This has a low rate of management and is easy to implement but it costs money when the driver has nothing to do. Combining multiple restaurants with one delivery fleet and managing the fleet via a routing and timing algorithm is rather new. The time component could be implemented by using a static time for every order or a back channel when the order is placed. The first solution does not give enough time gain while the second requires a lot of infrastructure. This is why a custom approach has to be made in order to optimize the time part.
\section{Approach and Forecasting Basics}\label{section:Approach and Basics}
After searching for fitting materials, the book "Forecasting: Principles and Practice" by R. Hyndman and A. Athanasopoulos\cite{Hyndman.2013} was chosen as a starting point, since it covers the topic pretty good for forecasting beginners. After reading the book it was determined to follow the steps the book suggests to create the forecast. The book had five basic steps for forecasting which will be explained in the following.\newline
There are various Stakeholders, e.g. the people who provide the data, those in charge of the use of the forecast and those who monitor how the forecast is put together. All have different views on the problem, different needs and different preferences. In the first step, these Stakeholders are consulted to define the problem.  who it is required and how it fits into the process of the ones who required it have to be asked. Also the people who support the forecast with collecting data, maintaining the data and use the forecast for future planning have to be figured out. After the problem is defined, in a second step, all information available is gathered. With the variety of restaurants, meals and levels of utilization of staff during the day, it is to be expected there will not be enough statistical data. In order to make up for the little and imperfect data, the expertise of the people who collected this data and use the forecasts is also taken into account for the forecast. When the data collecting is finished, a preliminary analysis is done to identify patterns and abnormalities. This is done before fitting models to the data in order to prevent false assumptions. For the analysis the data is put into a graph to get a visual result. In the graph abnormalities like wrong data, patterns, e.g. trends, seasonalities or business cycles, can be detected without much effort and outliers can be questioned before beginning the process. Now the real forecasting can begin by choosing adequate models. These models are fitted to the data and expertise and the forecasting is done. The result of the different models are finally compared in regards to their prediction versus the actual events.\newline
Now that there is an overall plan for the forecasting, a closer look is taken on the underlying parts of the forecast.

\subsection{Time Series}\label{subsection:Time Series}
Since the events are happening on a time line, time series decomposition was the choice. Time series can have different patterns that is why putting the data in a graph first is very useful. The book and the slides likewise addressed common patterns and methods which can be used on time series and improve forecasts.
The time series can then be split into different parts to support the corresponding pattern. There are three types of time series patterns. The first pattern is the trend. The trend is an decrease or increase over a long period of time. The second second is the season pattern. It is influenced by seasonal factors and has a known and fixed time period and. The third one is the cyclic pattern. Cyclic pattern influence usually has fluctuations of at least 2 years and the period is not fixed. The available data should be transformed to a time series and divided into suitable components, in case patterns are present.
\subsection{Algorithms for Forecasting}\label{subsection:Algorithms for Forecasting}
Now that the time series is investigated for patterns, it is time to forecast. The book and slides offer different algorithms to fulfill this task. Three algorithms are chosen to be presented more closely.
\subsubsection{Moving Average}\label{subsubsection:Moving Average}
A classical method is the moving average. It is used to calculate the next forecast value of a time series iteratively. For the time series it is done by taking all prior events in account for the next value. The calculated values are independent. The formula for the average at time t is: m(t)= 1/n E (n-1, 0) x(t-1)
\subsubsection{Weighted Moving Average}\label{subsubsection:Weighted Moving Average}
The window which is taken into the calculation of the next value can also be weighted. In contrast to the normal Moving Average, a window for the amount of values included in the creation of the average value. The window moves with the time series. When a calculating the next value, the first value is removed from the window. This way only a specific amount of values stays in the calculation.
\subsubsection{Simple Exponential Smoothing}\label{subsubsection:Simple Exponential Smoothing}

Cite papers with \verb+\cite{BibTeX key}+\\
A comprehensive introduction to Operations Research:
