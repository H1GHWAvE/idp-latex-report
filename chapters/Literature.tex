\renewcommand{\thepage}{\arabic{page}}
\chapter{Review of Literature and Research}\label{chapter:Review of Literature and Research}
This chapter is about the sources and the information gained before starting the actual forecasting.
\section{Research}\label{section:Research}
In order to get a good overview over the problem, a search in Google Scholar was done. The goal was to find similar problems and approaches to it. Different words were chosen for the search, like "preparation time", "meal", "restaurant" and "forecast". The results did not give the right output for this problem. Most of the forecasting in restaurant was about the amount of staff needed, the amount of meals which will be sold over a period or the size a buffet has to
be. These topics did not fit the problem given because the problem stated is pretty unique. There are not many companies who do last mile delivery with time critical materials. Very often the restaurant has its own fleet of driver which is available all the time and they send it when a meal has to be delivered. This has a low rate of management and is easy to implement but it costs money when the driver has nothing to do. Combining multiple restaurants with one delivery fleet and managing the fleet via a routing and timing algorithm is rather new. The time component could be implemented by using a static time for every order or a back channel when the order is placed. The first solution does not give enough time gain while the second requires a lot of infrastructure. This is why a custom approach has to be made in order to optimize the time part.
\section{Approach and Forecasting Basics}\label{section:Approach and Basics}
After searching for fitting materials, the book “Forecasting: Principles and Practice” by R. Hyndman and A. Athanasopoulos was chosen as a starting point, since it covers the topic of forecasting in general pretty good for starting the job. After reading the book, it was determined to follow the steps the book suggests to create the forecast. The book had five basic steps for forecasting which will be explained in the following\ref{Hyndman.2015}:

\begin{enumerate}
\item Problem Definition
\item Gathering Information
\item Preliminary Analysis
\item Choosing and Fitting Models
\item Using and Evaluating the Forecast
\end{enumerate}

There are various Stakeholders, e.g. the people who provide the data, those in charge of the use of the forecast and those who monitor how the forecast is put together. All have different views on the problem, different needs and different preferences. In a first step, the process is mapped out and the Stakeholders and their involvement in the process is determined. They are being consulted on their view of the problem and their needs.\newline
In the second step, the data available for the process is determined. (benutzt Du SIPOC? Stakeholder, Input, Process, Output, Customer?) With the variety of restaurants, meals and levels of utilization of staff during the day, it is to be expected there will not be enough statistical data for all scenarios. In order to make up for the little and imperfect data, the expertise of the people who collected this data and use the forecasts is also taken into account for the forecast. When the initial? data collection collecting is finished, a preliminary analysis is done to identify patterns and abnormalities. This is done Du beschreibst keinen linearen 7 Step process mit before. Schreib einfach Next, models are fit… … und ich bin nicht sicher ob die models false assumptions vermeiden sollen. before fitting models to the data in order to prevent false assumptions. For the analysis the data is put into a graph to get a visual result. In the graph abnormalities like wrong data, patterns, e.g. trends, seasonalities or business cycles, can be detected without much effort and outliers can be questioned before beginning the process das hast du doch schon im vorletzen Schritt getan. Bei welchem Schritt bist Du jetzt?. Now the real forecasting can begin by choosing adequate models (nochmal models?). These models are fitted to the data and expertise and the forecasting is done. The result of the different models are finally compared in regards to their prediction versus the actual events.
Now that there is an overall plan for the forecasting, a closer look is taken on the underlying parts of the forecast.
Diesen 7 schritt prozess musst Du noch klarer strukturieren und klarer machen, was warum getan wird. Ausserdem ist nicht klar, was der output der 7 schritte ist. Ich dachte es waere ein Forecast, aber im Abschnitt ALGORITHMS gehst Du zurueck und waehlst nochmal (zum 3. mal?) ein Model(=Algorithm) aus.

\begin{table}[h]
\centering
\caption{SIPOC Diagram derived from the five basic steps}
\label{sipoc}
\scalebox{0.75}{%
\begin{tabular}{|l|l|l|l|l|}
\hline
Supplier                                                                                                                                                                       & Input                                                                              & Process                                                                                                                    & Output                                                                 & Customer                                                                                       \\ \hline
\begin{tabular}[c]{@{}l@{}}Emanuel Pallua (COO)\\ Stefan Rothlehner (CTO)\\ Sergej Krauze (CTO)\\ Sebastian Sontheimer(BI)\\ Operations Team\end{tabular} & \begin{tabular}[c]{@{}l@{}}Knowledge\\ Experience\\ Historic Database\end{tabular} & \begin{tabular}[c]{@{}l@{}}Preliminary Analysis\\ Choosing Models\\ Fitting Models\\ Forecasting\\ Evaluating\end{tabular} & \begin{tabular}[c]{@{}l@{}}Forecast\\ Evaluation\\ Result\end{tabular} & \begin{tabular}[c]{@{}l@{}}Emanuel Pallua\\ Sebastian Sontheimer\\ Operation Team\end{tabular} \\ \hline
\end{tabular}}
\end{table}
\subsection{Time Series}\label{subsection:Time Series}
The method of time series puts events (forecasted or actual) onto a time axis. Charting this as a graph often reveals patterns. These patterns are divided into 3 different kinds \cite{Hyndman.2013}. The first pattern is the trend. The trend is an decrease or increase over a long period of time. The second second is the season pattern. It is influenced by seasonal factors and has a known and fixed time period and. The third one is the cyclic pattern. Cyclic pattern influence usually has fluctuations of at least 2 years and the period is not fixed. Since the events at VOLO are happening on a time line, time series decomposition was the choice. The available data is transformed to a time series and divided into suitable components, in case patterns are present.\newline
The time series are used in the preliminary analysis to identify abnormalities and patterns. They can also be used for better understand of the forecast results but most of the time only the numbers matter.
\subsection{Algorithms for Forecasting}\label{subsection:Algorithms for Forecasting}
Having charted the time series of actual events and analysed the patterns, it is not time to forecast for the first preparation times.. The book offers different algorithms to fulfill this task from which three are chosen und presented.
\subsubsection{Moving Average}\label{subsubsection:Moving Average}
A classical method is the moving average. It is used to iteratively calculate the next forecast value of a time series. The next values is generated by taking all prior events in account. The calculated values are independent of each other. The formula to calculate the forecast for the event at point t is:
\begin{center}
\begin{equation}
ma_{(t)}= \frac{1}{t}\sum^{t-1}_0 x_{(t-1)}
\end{equation}
\end{center}
Same equation also applies to orders.

\subsubsection{Weighted Moving Average}\label{subsubsection:Weighted Moving Average}
Instead of taking all events (as in the moving average), weighted moving average defines a window to be used, i.e. only the last x events or time frame. This window moves according to the current position on the time series. When calculating the next forecast value, the first value is removed from the window respectively when moving to the next day the first day of the frame is removed from the calculation. This way only a specific amount of orders or days stays in the calculation. The weighted moving average for point t is calculated as following:

\begin{center}
\begin{equation}
wma_{(t)}= \frac{1}{n}\sum^{t-1}_{t-n-1} x_{(t-1)}
\end{equation}
\end{center}
Same equation also applies to orders.

\subsubsection{Simple Exponential Smoothing}\label{subsubsection:Simple Exponential Smoothing}
Another approach is the Simple Exponential Smoothing which return a smoothed forecast. A weighted average is calculated from the occurances before which is weighted by the factor \alpha with the constraint 0 {\le} {\alpha} {\le} 1. {\alpha} specifies the ratio between the weight of the last order's preparation time and the forecast of the values before that. The function is recursive and has no limit for the number of predecessors. The only thing which has to be defined is the starting value ses_{start}. Forecasting the event at point t is as following:

\begin{center}
\begin{equation}
ses_1=\alpha*x_{0}+(1-\alpha)*ses_{start}
\end{equation}
\begin{equation}
ses_t=\alpha*x_{t-1}+(1-\alpha)*ses_{t-1}
\end{equation}
\end{center}
