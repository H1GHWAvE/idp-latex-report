\renewcommand{\thepage}{\arabic{page}}
\chapter{Review of Literature and Research}\label{chapter:Review of Literature and Research}
This chapter is about the sources and the information gained before starting the actual forecasting.
\section{Research}\label{section:Research}
In order to get a good overview over the problem, a search in Google Scholar was done. The goal was to find similar problems and approaches to it. Different words were chosen for the search, like "preparation time", "meal", "restaurant" and "forecast". The results did not give the right output for this problem. Most of the forecasting in restaurants was about the amount of staff needed, the amount of meals which will be sold over a period or the size a buffet has to
be. These topics did not fit the problem given as it is pretty unique. There are not many companies who do last mile delivery with time critical materials, as VOLO does. Typically the restaurant has its own fleet of drivers. They are available all the time and sent when someone's ordered meal is ready to be delivered. This needs no sophisticated management  and is easy to implement but wastes money when the driver has nothing to do. Combining multiple restaurants with one delivery fleet and managing the fleet via a routing and timing algorithm is rather new. The time component can be implemented by using a static time for every order or a back channel when the order is placed. A back channel would be a reply message by the restaurant containing the time the meal will be ready. The first solution does not give enough time gain while the second requires a lot of infrastructure. This is why a custom approach has to be made in order to optimize the time part.
\section{Approach}\label{section:Approach and Basics}
After searching for fitting materials, the book "Forecasting: Principles and Practice" by R. Hyndman and A. Athanasopoulos was chosen as a starting point, since it covers the topic of forecasting in general pretty good for starting the job. After reading the book, it was determined to follow the steps the book suggests to create the forecast. The book (\cite{Hyndman.2013}) has five basic steps for forecasting which will be explained in the following:

\begin{enumerate}
\item Problem Definition
\item Gathering Information
\item Preliminary Analysis
\item Choosing and Fitting Models
\item Using and Evaluating the Forecast
\end{enumerate}

In a first step, the process of the forecast is mapped out and the Stakeholders and their contribution to the forecast are determined. They are being consulted on their view of the problem and their needs. For this purpose a SIPOC diagram is created (\cite{SIPOC}). With this information a problem statement is defined.\newline
In the second step, the data available for the process is determined. With the variety of restaurants, meals and levels of utilization of staff during the day, it is to be expected there will not be enough statistical data for all scenarios. In order to make up for the little and imperfect data, the expertise of the people who collected this data and who use the forecasts is also taken into account.\newline
The third step is the preliminary analysis. It is done to identify patterns, abnormalities and get an overview over the data. Also historic average preparation times are calculated to have a rough estimation for the outcomes of the forecast. The data is put into graphs to have a visual output and to detect invalid inputs, patterns or trends\newline
The fourth step is to choose good models and to fit the processed data set to these models. A model is the way the data is forecast, e.g. the granularity or algorithm used.\newline
The fifth and last step is all about using the models and getting results. The results are then evaluated and discussed how they will fit the process.
\section{Forecasting Basics}\label{section:Forecast Basics}
In order to establish a foundation for the forecast, certain key terms are specified.
\subsection{Time Series}\label{subsection:Time Series}
Time series are used on events that happen sequentially over time. This method puts events (forecasted or actual) onto a time axis.
They are used in the preliminary analysis to identify abnormalities and patterns. They will also be used to gain a better understanding of the forecast results in addition to the pure numbers. Charting data as a graph often reveals patterns. These patterns are divided into 3 different kinds (\cite{Hyndman.2013}). The first pattern is the trend pattern. A trend is a decrease or increase over a long period of time. The second pattern is the season pattern. Seasonal patterns appear when data is influenced by seasonal factors and the effect has a known and fixed time period. The third one is the cyclic pattern. Cyclic patterns at least fluctuate over 2 years and have no fixed period of recurrence. An example is a rapid increase followed by a slow decrease, which is not happening every summer, but from time to time.\newline
Orders are discrete events over time well suited to be shown in time series, so time series decomposition was the choice. The available data is transformed to a time series and divided into suitable components, hopefully revealing patterns.
\subsection{Algorithms for Forecasting}\label{subsection:Algorithms for Forecasting}
The book (\cite{Hyndman.2013}) offers different algorithms to forecast on time series. Three are chosen und presented.
\subsubsection{Moving Average}\label{subsubsection:Moving Average}
A classical method is the Moving Average. It is used to iteratively calculate the next forecast value of a time series. The next value is generated by taking all prior events in account. The calculated values are independent of each other.\newline
The formula to calculate the forecast for the order t is:
\begin{center}
\begin{equation}
ma_{(t)}= \frac{1}{t}\sum^{t-1}_{l = 0} x_{l}
\end{equation}
\end{center}

\subsubsection{Weighted Moving Average}\label{subsubsection:Weighted Moving Average}
Instead of taking all events (as in the Moving Average), Weighted Moving Average defines a weight, i.e. only the last x events or a time frame  as a moving window, to be used for the forecast. This window moves according to the current position on the time series. The window can be specified in terms of number of orders or time units. When using orders, $n$ equals the number of orders in the window. When using a time specification, $n$ is the number of orders which is in the specified time frame, e.g. if the window is three days, day one has 10 orders, day two 5 and day 3 has 2 orders, the resulting $n$ is 18. When calculating the next forecast value, older values leave the window and new ones enter it. For instance, when moving to a new day, the time frame moves on, dropping the all orders of the day from the beginning. This way, only a specific amount of time units, e.g. days, and the associated orders stay in the calculation. The size of the window is called weight. When creating models, different weight sizes can be chosen.\newline
The Weighted Moving Average for order t is calculated as following:

\begin{center}
\begin{equation}
wma_{(t)}= \frac{1}{n}\sum^{t-1}_{l = t-n-1} x_{l}
\end{equation}
\end{center}

\subsubsection{Simple Exponential Smoothing}\label{subsubsection:Simple Exponential Smoothing}
Another approach is the Simple Exponential Smoothing which returns a smoothed forecast. The forecast value is calculated recursively from all occurrences before. The two components to forecast the next value, last order's preparation time and the calculated forecast for it, are weighted by the factor $\alpha$ with the constraint $0$ $\le$ $\alpha$ $\le$ $1$. The $\alpha$ specifies the ratio between these two. The only thing which has to be defined is the starting value $ses_{start}$.\newline
Forecasting the value for event t is as following:

\begin{equation}
  \begin{align}
  	ses_1 &= \alpha * x_{0} + (1 - \alpha) * ses_{start} \\
  	ses_t&=\alpha*x_{t-1}+(1-\alpha)*ses_{t-1}
  \end{align}
\end{equation}


\subsubsection{Evaluation Criteria}
In order to compare different approaches, the root mean square error (RMSE) is used. It represents sample standard deviation of the actually recorded time from the forecasted value. It sums up the squared error for each calculation and divides the result by the number of calculations.\newline
The square rooted result is the RMSE.
\begin{center}
\begin{equation}
RMSE = \sqrt{\frac{\sum^{n}_{x=0}{(y_{forecast} - y_{actual})^{2}}}{x}}
\end{equation}
\end{center}
