\chapter{Introduction}\label{chapter:Introduction}

\renewcommand{\thepage}{\arabic{page}}
\setcounter{page}{1}

The food delivery market is a fast growing market with an giant volume. Rocket internet predicts the market to have a value of 90 billion Euro by 2019. CITE!! This money attracts many companies which fight for the supremacy in the market. In order to achieve this goal they have to differentiate from their competitors. Some do this by being the cheapest, others have the biggest variety of restaurants they offer and still others try to optimize the process from ground up.\newline
An example is is the Munich based startup called VOLO. The idea of VOLO is to provide a great and effective experience in food delivery. Starting with premium restaurants with an appealing shop, followed by an process optimized driver fleet for the restaurants and fast deliveries. The optimized and fast deliveries are backed by an underlying algorithm. The algorithm is based on the travelling salesman algorithm and calculates the best solution for a number of drivers who have to fulfill a number of deliveries. It assigns each driver a route and deliveries he has to fulfill including the route the driver should take and the time needed to complete the task. This enables drivers have less empty drives and to minimize idle times from which the company and the driver profits. The driver is able to earn more money from loaded times and tips and the company has to pay few idle times where the driver is not used.\newline
This algorithm is perfect from the point in time the driver instantly picks up the food, leaves the restaurant and starts driving. The only problem is that the food is almost never ready at the moment the algorithm sends the driver to the restaurant. It is crucial for a food delivery to pick the meal up at the exactly right time. On the one hand being late is bad, because if the customer gets his food cold or too late he is likely not to order again from the company. On the other hand when the driver is too early at the restaurant he has to wait for the order to be prepared. In this case the time gain is lost and with it the advantage over the other competitors.
This is why VOLO has the need to create a forecast which tells the algorithm the forecasted preparation time for the food so the driver can be send to the restaurant just in time.\newline
The following chapters will explain the proceeding to solve the problem. It will start with the resources used to generate knowledge, then focus on the methodology used to forecast and it will finish with evaluating the result and a conclusion.
