\chapter{Introduction}\label{chapter:Introduction}

\renewcommand{\thepage}{\arabic{page}}
\setcounter{page}{1}

The food delivery market is a fast growing market with an giant volume. Rocket internet predicts the market to have a value of 90 billion Euro by 2019. This money attracts many companies which fight for the supremacy in the market. In order to achieve this goal they have to differentiate from their competitors. Some do this by being the cheapest, others have the biggest variety of lower quality restaurants to offer and still others try to optimize the delivery process to improve the quality of service.\newline
An example is is the Munich based startup called VOLO. The idea of VOLO is to provide a great and effective experience in food delivery. Starting with premium restaurants with an online appealing shop, followed by process optimized driver fleet. The optimized fleet is build on the underlying algorithm which makes deliveries as fast as possible while being able to serve many orders at once. The algorithm is based on the travelling salesman algorithm and calculates the best solution for a number of drivers who have to fulfill a number of deliveries. It assigns each driver deliveries she has to fulfill as well as the best route and order to pick up and delivery orders.. This minimizes empty drives and idle time for the benefit of the driver and the company. The driver is able to earn more money from loaded times and tips and VOLO has to pay less for drivers since the idle times are reduced to a minimum.\newline
This algorithm works optimal from the point in time when the driver receives the order information and starts driving to the restaurant as well as when she leaves the restaurant and starts driving to the customer. The problem is that the food is almost never ready at the time the algorithm sends the driver to the restaurant. It is crucial to pick up the meal at the exact right time. Being late is bad, as the food will be cold or no longer fresh. Being early is bad, as it induces waiting time for the driver as she is early at the restaurant and has to wait for the order to be prepared. Forecasting the arrival time gives an advantage over the competitors.\newline
This is why VOLO has the need to create a forecast predicting preparation time for the food so the driver can be send to the restaurant just in time.\newline\newline
The following chapters will explain the proceeding to solve the problem. It will start with the resources used to generate knowledge, then focus on the methodology used to forecast and it will finish with evaluating the result and a conclusion.
