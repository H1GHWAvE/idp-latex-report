\chapter{Introduction}\label{chapter:Introduction}

\renewcommand{\thepage}{\arabic{page}}
\setcounter{page}{1}

The food delivery market is a fast growing market with an giant volume. Rocket Internet predicts it to have a value of 90 billion Euro by 2019 (\cite{Rocket}). This money attracts many companies fighting for the supremacy. In order to achieve this goal, they have to differentiate from their competitors. Some try to be the cheapest, others have the biggest variety of lower quality restaurants to offer and still others try to excel in their delivery by optimizing the delivery process and quality of service.\newline
An example is the startup VOLO, originally founded in Munich. The idea of VOLO is to provide a fast and efficient food delivery, starting with premium restaurants, an appealing online shop and topped with a fast and efficient driver fleet. The fleet is relying on an algorithm making the deliveries itself as fast as possible while allowing one driver to serve many orders at once. The algorithm is based on the travelling salesman algorithm and calculates the best routing solution for n drivers and m deliveries. A driver is assigned a route and sequence to pick up and deliver one or more orders. This minimizes empty drives and idle time for the benefit of the driver and the company. The driver is able to earn more money from loaded times and tips and VOLO has to pay less for drivers since the idle times are reduced to a minimum.\newline
This algorithm works optimal from the point in time when the driver receives the order information and starts driving to the restaurant and then again when she leaves the restaurant and starts driving to the customer. The problem is that the food is almost never ready at the time the algorithm sends the driver to the restaurant. It is crucial to pick up the meal at the exact right time. Being late is bad, as the food will be cold or no longer fresh. Being early is bad, as it induces waiting time for the driver as she has to wait for the order to be finished. Forecasting the optimal arrival time gives an advantage over the competitors.\newline
This is why VOLO has the need to create a forecast, predicting preparation time for the food so the driver can be sent to the restaurant just in time.\newline
The following chapters will explain the proceeding to solve the problem. It will start with the resources used to generate knowledge, then focus on the methodology used to forecast and it will finish with evaluating the result and a conclusion.
