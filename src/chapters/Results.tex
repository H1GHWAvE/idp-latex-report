\chapter{Results}\label{chapter:Results}
Now that the models are set and run, their RMSE is calculated. The lower the error the more accurate the model. This is done for the training as well as the test set and the errors are compared and their difference are shown. When the difference between the two sets is low, the predictive relationships of the test set are strong.\newline
In order to see whether the forecasting model is an improvement according to the requirements, it is compared with the basic approach of the Operation Team.\newline
The results are presented following the enumeration of time categories done in Chapter \ref{subsection:Categorizing by Order}. The forecast is done one time with the time categorization and no slots and one time with slots.\newline
Each table represents a group of models with the same time categorization and use of slots. The algorithm used for the model is shown in the first column. Then follow 2 groups of 3 columns for specific restaurant (yum2take) and all restaurants. These groups show the RMSE for training and Test Data and the difference in their RMSE. A negative difference indicates a more accurate result for the test set and vice versa. The difference is not significant as long as it is below 5\%.\newline

The Weighted Moving Average can be calculated using different weights, depending on the volume of data available. When calculating the Simple Exponential Smoothing, different $\alpha$ are used. Values for Alpha are stepped up from 0.1 to 0.9 in .1 steps and values for the forecasts calculated. Only the best results are shown in the table.
\section{No time categorization}\label{section:No time categorization}
These are the results for the models which are using no time categorization.
\subsection{Without slot categorization}
Table \ref{tab:No time categorization results} shows the results for the models which use no time categorization and no slots.
\begin{table}[h]
\centering
\caption{No time and slot categorization}
\label{tab:No time categorization results}
\scalebox{0.5}{%
\begin{tabular}{lllllll}
\hline
       & \multicolumn{3}{l}{yum2take}                                                 & \multicolumn{3}{l}{all}                                                    \\ \hline
Algorithm   & \begin{tabular}[c]{@{}l@{}}training\\ \#686\end{tabular} & \begin{tabular}[c]{@{}l@{}}test\\ \#945\end{tabular} & difference & \begin{tabular}[c]{@{}l@{}}training\\ \#2129\end{tabular} & \begin{tabular}[c]{@{}l@{}}test\\ \#3196\end{tabular} & difference \\ \hline
MA      & 6.02755                         & 5.950609                       & -0.07694  & 8.409876                         & 8.565466                       & 0.15559  \\ \hline
MA (650/2125) & 5.738142                         & 5.880966                       & 0.142823  & 4.542798                         & 8.976687                       & 4.433889  \\ \hline
SES (0.1/0.1) & 6.116109                         & 5.973192                       & -0.14291  & \multicolumn{1}{l}{8.560794}               & \multicolumn{1}{l}{8.65649}             & 0.095695  \\ \hline
\end{tabular}}
\end{table}
\newline\newline\textbf{No restaurant categorization}\newline
The RMSE for the Moving Average model for this categorization is 8.4 minutes. When comparing the training to the test set the accuracy decreases by 0.15 minutes. The Weighted Moving Average, the second group of models, has different models with different weights which range from 25 orders to 2125 orders. Most of the resulting RMSE are between 7.9 minutes and 8.5 minutes. There is one outlier with the weight of 2100 orders and 2125 orders and 6.7 minutes and 4.5 minutes. All models lose accuracy when compared to their test sets especially the outliers. The results of the outliers show that there is only a weak predictive relationship between the training set and the test set.\newline The Simple Exponential Smoothing has the best result with 8.6 minutes for an $\alpha$ of 0.1 and rises up to 11.8 minutes for an $\alpha$ of 1.
Except for the outliers, the biggest improvement over the basic approach is around 7\% when using the Weighted Moving Average with a weight of 1850.
\newline\newline\textbf{Restaurant categorization using yum2take}\newline
The forecasts results in a RMSE of 6 minutes for the Moving Average model. The Weighted Moving Average is calculated in different weights. The weights start with 25 orders and increase step by step to 675 orders. The RMSE of these forecasts is stable between 6 minutes and 6.5 minutes and the best result is achieved with a weight of 675 orders. The best RMSE for the Simple Exponential Smoothing is the result of an $\alpha$ of 0.1. When increasing $\alpha$ the RMSE also increases to up to 8.5 minutes of RMSE. The accuracy improves for all models when using the test set by 0.07 minutes for the Moving Average, between 0.07 minutes up to 0.45 minutes when using the weighted version and between 0.14 minutes and 0.42 minutes for the Simple Exponential Smoothing.\newline
When using the restaurant categorization for this time categorization, an improvement up to 30\% for the Moving Average can be observed.

\subsection{With slot categorization}
The results of the models without time categorization but with slots are shown in Table \ref{tab:No time but slot categorization results}.
\begin{table}[h]
\centering
\caption{No time but slot categorization}
\label{tab:No time but slot categorization results}
\scalebox{0.5}{%
\begin{tabular}{lllllll}
\hline
       & \multicolumn{3}{l}{yum2take}                                                 & \multicolumn{3}{l}{all}                                                    \\ \hline
Algorithm  & \begin{tabular}[c]{@{}l@{}}training\\ \#686\end{tabular} & \begin{tabular}[c]{@{}l@{}}test\\ \#945\end{tabular} & difference & \begin{tabular}[c]{@{}l@{}}training\\ \#2129\end{tabular} & \begin{tabular}[c]{@{}l@{}}test\\ \#3196\end{tabular} & difference \\ \hline
MA      & 5.980634                         & 5.943105                       & -0.03752  & 8.203962                         & 8.514927                       & 0.310964  \\ \hline
MA (25/100) & 6.165134                         & 6.017991                       & -0.14714  & 7.957781                         & 8.492864                       & 0.535082  \\ \hline
SES (0.1/0.1) & 6.065397                         & 5.982731                       & -0.08266  & 8.379257                         & 8.618189                       & 8.618189  \\ \hline
\end{tabular}}
\end{table}

\newpage\textbf{No restaurant categorization}\newline
Slot categorizations has an RMSE of 8.2 minutes for the Moving Average case. The Weighted Moving Average is about the same with some results around 8 minutes (e.g. weight of 100 orders, 7.95 minutes). The weights for this algorithm are from 25 orders up to 225 orders in 25 orders steps. This categorization models with the Simple Exponential Smoothing algorithm start from 8.4 minutes and an $\alpha$ of 0.1 and continue to 11.3 minutes when $\alpha$ reaches 1. For all models the accuracy decreases when the test set is used. They decline between 0.25 minutes and 0.54 minutes.\newline
The best result for this time categorization has the Weighted Moving Average with a weight of 100. An error of 7.95 minutes results in an improvement of around 6\%.
\newline\newline\textbf{Restaurant categorization using yum2take}\newline
The Moving Average for yum2take results in a RMSE of little under 6 minutes. The weighted average is used with 25 and 50 order steps from which the 25 orders scores better with 6.2 minutes while 50 orders has 6.3 minutes as RMSE. The Simple Exponential Smoothing also gives good results with close to 6 minutes for an $\alpha$ of 0.1. When increasing $\alpha$ the RMSE grows to almost 8 minutes. The accuracy between training and test set is almost identical for the Moving Average algorithm, improves by 0.15 minutes for the Weighted Moving Average and also improves for all Simple Exponential Smoothing models with a range from 0.08 minutes up to 0.47 minutes while decreasing the accuracy.\newline
When using the Moving Average an improvement of almost 30\% can be achieved in contrast to the basic approach.
\section{Day categorization}\label{section:Day categorization}
These are the models using day categorization and their results.
\subsection{Without slot categorization}
The results in Table \ref{tab:Day categorization} are generated for categorization by day. There is no Weighted Moving Average model since there has not been enough data to do forecast in a meaningful manner.
\begin{table}[h]
\centering
\caption{Day categorization without slots}
\label{tab:Day categorization}
\scalebox{0.5}{%
\begin{tabular}{lllllll}
\hline
       & \multicolumn{3}{l}{yum2take}                                                 & \multicolumn{3}{l}{all}                                                    \\ \hline
Algorithm  & \begin{tabular}[c]{@{}l@{}}training\\ \#686\end{tabular} & \begin{tabular}[c]{@{}l@{}}test\\ \#945\end{tabular} & difference & \begin{tabular}[c]{@{}l@{}}training\\ \#2129\end{tabular} & \begin{tabular}[c]{@{}l@{}}test\\ \#3196\end{tabular} & difference \\ \hline
MA      & 6.011955                         & 5.993232                       & -0.01872  & 8.403266                         & 8.57489                        & 0.171623  \\ \hline
SES (0.1/0.1) & 6.080652                         & 5.995665                       & -0.08498  & 8.434747                         & 8.57399                        & 0.139242      \\ \hline
\end{tabular}}
\end{table}
\newpage\textbf{No restaurant categorization}\newline
The Moving Average model returns a RMSE of 8.4 minutes while the Simple Exponential Smoothing models start with 8.4 minutes for the $\alpha$ of 0.1 and end with 9.8 minutes for the $\alpha$ of 1. The test set is slightly better than the trainings set with 0.17 minutes for the Moving Average model while the $\alpha$ of 0.1 has the worst decline with 0.13 for the Simple Exponential Smoothing and the other values increase accuracy up to the best improvement of almost 1 minute for the $\alpha$ of 1.\newline
When comparing these results with the basic approach an improvement of only 1\% can be achieved by using this time categorization.
\newline\newline\textbf{Restaurant categorization using yum2take}\newline
When using day and slot categorization the Moving Average performs well again, compared to the other models, with an RMSE of 6 minutes. The result improves for the test set by 0.08 minutes. Simple Exponential Smoothing returned 6.1 minutes for an $\alpha$ of 0.1 and only slowly increased to 6.9 minutes for the maximum of $\alpha$ of 1. The accuracy also improved for up to 0.4 minutes from the training to the test set.\newline
When using restaurant categorization and yum2take, an improvement of 30\% for the RMSE can be seen when using Moving Average.
\subsection{With slot categorization}
When using day categorization and slots Table \ref{tab:Day and slot categorization results} is the result.
\begin{table}[h]
\centering
\caption{Day categorization with slots}
\label{tab:Day and slot categorization results}
\scalebox{0.5}{%
\begin{tabular}{lllllll}
\hline
       & \multicolumn{3}{l}{yum2take}                                                 & \multicolumn{3}{l}{all}                                                    \\ \hline
Algorithm  & \begin{tabular}[c]{@{}l@{}}training\\ \#686\end{tabular} & \begin{tabular}[c]{@{}l@{}}test\\ \#945\end{tabular} & difference & \begin{tabular}[c]{@{}l@{}}training\\ \#2129\end{tabular} & \begin{tabular}[c]{@{}l@{}}test\\ \#3196\end{tabular} & difference \\ \hline
MA      & 6.011955                         & 5.993232                       & -0.01872  & 8.162479                         & 8.535424                       & 0.372944  \\ \hline
MA (15/60)  & 6.127325                         & 6.109218                       & -0.0181  & 7.859824                         & 8.737695                       & 0.87787  \\ \hline
SES (0.1/0.1) & 6.04016                         & 6.104262                       & 0.064101  & 8.220405                         & 8.581225                       & 0.360819  \\ \hline
\end{tabular}}
\end{table}
\newline\newline\textbf{No restaurant categorization}\newline
The RMSE for the Moving Average model is around 8.2 minutes while the Weighted Moving Average range from 7.9 minutes, for the best case and a weight of 60 days, to 8.3 minutes for the weight of 15 days. The Simple Exponential Smoothing gets worse the bigger $\alpha$ becomes. It starts with 8.2 minutes for 0.1 and rises to 10.9 minutes for an $\alpha$ of 1. When comparing Training to Test Set, the accuracy declines for all Moving Average variants. The Moving Average has a decline in accuracy of 0.37 minutes as well as the the weighted inaccuracy grows from 0.24 minutes to 0.88 minutes the bigger the weight gets. For the Simple Exponential Smoothing there is only decline for the $\alpha$ from 0.1 to 0.3, 0.36 minutes, 0.26 minutes and 0.12 minutes. The remaining $\alpha$ steadily improve the accuracy up to 1.28 minutes for the value of 1.\newline
Using the Weighted Moving Average of 60 orders results in an improvement of 7\%.
\newline\newline\textbf{Restaurant categorization using yum2take}\newline
For yum2take the results are better than for the forecast which uses all orders. The Moving Average has the best result with about 6 minutes. When weighted it increases to 6.1 minutes, for both weights, namely 15 and 30 orders, which is not much difference to the training set result. Same goes for the Simple Exponential Smoothing, where the RMSE and the difference between the sets grows the bigger $\alpha$ gets. The difference between training set is for all models neglectable because it is around 0.5 minutes, which is too small to have an impact on the real life scenario.\newline
For restaurant categorization and yum2take an improvement of around 30\% over the basic approach can be achieved.
\section{Week categorization}\label{section:Week categorization}
The following results are generated by the models using week categorization for the time category.
\subsection{Without slot categorization}
The results of categorization by week are shown in Table \ref{tab:Week categorization result}.
\begin{table}[h]
\centering
\caption{Week categorization without slots}
\label{tab:Week categorization result}
\scalebox{0.5}{%
\begin{tabular}{lllllll}
\hline
       & \multicolumn{3}{l}{yum2take}                                                 & \multicolumn{3}{l}{all}                                                    \\ \hline
Algorithm   & \begin{tabular}[c]{@{}l@{}}training\\ \#686\end{tabular} & \begin{tabular}[c]{@{}l@{}}test\\ \#945\end{tabular} & difference & \begin{tabular}[c]{@{}l@{}}training\\ \#2129\end{tabular} & \begin{tabular}[c]{@{}l@{}}test\\ \#3196\end{tabular} & difference \\ \hline
MA      & 6.074067                         & 5.967819                       & -0.10624  & 8.402317                         & 8.568181                       & 0.165863  \\ \hline
MA (8/16)   & 6.120197                         & 5.949482                       & -0.17071  & 8.164492                         & 8.814434                       &      \\ \hline
SES (0.2/0.2) & 6.008943                         & 5.981932                       & -0.02701  & 8.399327                         & 8.580201                       & 0.180873  \\ \hline
\end{tabular}}
\end{table}
\newline\newline\textbf{No restaurant categorization}\newline
The result for the Moving Average model has a RMSE of 8.4 minutes. When comparing the Training Data Set to the Test Data Set the accuracy lowers by 0.17 minutes. The Weighted Moving Average model with the weight of 4 orders is an outlier with 9.5 minutes. When using a weight of 8, 12 or 16 orders the RMSE lowers from 8.5 minutes to 8.2 minutes. Also when using the test set, the weight of 4 orders is still higher than the expected value with an RMSE of 8.9 minutes. In contrast to this outlier the other models only get more inaccurate, respectively 0.1 minutes, 0.29 minutes and 0.64 minutes for the three weights. The models with the Simple Exponential Smoothing algorithm have all similiar RMSE, ranging from 8.4 minutes to 8.55 minutes. When compared to the test set the accuracy gets slightly worse but not bigger than 0.21 minutes.\newline
Compared to the basic approach, an improvement of less than 4\% can be achieved by using the Weighted Moving Average.
\newpage\textbf{Restaurant categorization using yum2take}\newline
 The Moving Average model has a results with around 6.1 minutes RMSE. The Weighted Moving Average forecast models ranges from 6.1 to 6.6 minutes. The worst results come from the smallest (4 orders, 6.4 minutes) and greatest (16 orders, 6.6 minutes) while the weight of 8 and 12 orders has an RMSE of around 6.1 minutes. When using the Simple Exponential Smoothing, the results range from 6.0 minutes for an $\alpha$ of 0.2 and slowly grows up to 6.29 for an $\alpha$ of 1. The difference between training and Test Data Set is negative for most cases, meaning the result improves for the test set.\newline
 When using Moving Average, an improvement of 28\% over the basic approach can be achieved.
\subsection{With slot categorization}
In Table \ref{tab:Week and slot categorization results} the results for week and slot categorization are shown.
\begin{table}[h]
\centering
\caption{Week categorization with slots}
\label{tab:Week and slot categorization results}
\scalebox{0.5}{%
\begin{tabular}{lllllll}
\hline
       & \multicolumn{3}{l}{yum2take}                                                 & \multicolumn{3}{l}{all}                                                    \\ \hline
Algorithm   & \begin{tabular}[c]{@{}l@{}}training\\ \#686\end{tabular} & \begin{tabular}[c]{@{}l@{}}test\\ \#945\end{tabular} & difference & \begin{tabular}[c]{@{}l@{}}training\\ \#2129\end{tabular} & \begin{tabular}[c]{@{}l@{}}test\\ \#3196\end{tabular} & difference \\ \hline
MA      & 6.048462                         & 6.013698                       & -0.03476  & 8.21474                          & 8.532601                       & 0.31786  \\ \hline
MA (8/12)   & 6.137753                         & 5.952349                       & -0.1854  & 8.133761                         & 8.843515                       & 0.709754  \\ \hline
SES (0.3/0.3) & 5.935083                         & 6.155115                       & 0.220032  & 8.18856                          & 8.615477                       & 0.426916     \\ \hline
\end{tabular}}
\end{table}
\newline\newline\textbf{No restaurant categorization}\newline
For the Moving Average model a RMSE of 8.2 minutes was calculated. The different Weighted Moving Average models, weighted with 4, 8, 12 and 16 orders, are resulting in RMSE of 8.9 minutes, 8.19 minutes, 8.0 minutes, which is the best, and 8.1 minutes. Except for the weight of 4 orders, which increases accuracy by 0.04 minutes, the test set results are worse than the training set ones. The standard Moving Average model decreases accuracy by 0.32 minutes while the weight of 8 orders results in a decline of 0.31 minutes, 12 orders in 0.67 minutes and 16 orders weight in 0.71 minutes. The models using Simple Exponential Smoothing have all RMSE results of around 8.3 minutes, only an $\alpha$ between 0.2 and 0.6 have a RMSE of 8.2 minutes. The accuracy is lowered by around 0.4 minutes for all models when compared to the Test Data Set.\newline
When using this categorization and the Weighted Moving Average of 12 orders, the RMSE improves by almost 6\% over the basic approach.
\newline\newline\textbf{Restaurant categorization using yum2take}\newline
The forecast for the models including the categorization of week and slot returned 6.0 minutes RMSE for the Moving Average. The Weighted Moving Average can be calculated for steps of 4, 8 and 12 weeks as weight. It returns 6.3 minutes, 6.1 minutes and 6.2 minutes for the weights. These four results improve their accuracy when forecasting with the test set by between 0.03 and 0.13 minutes. For the Simple Exponential Smoothing an $\alpha$ between 0.2 and 0.5 gives good results of little under 6 minutes and growing for $\alpha$ over 0.5 to a RMSE of 6.45 minutes. The accuracy decreases for all values by about 0.1 to 0.2 minutes except for an $\alpha$ of 1 where it improves by 0.08 minutes.\newline
Compared to the basic approach, using this categorization and the Moving Average results in an improvement of almost 30\%.
\section{Weekday categorization}\label{section:Weekday categorization}
The following models use weekday categorization with and without slots.
\subsection{Without slot categorization}
Weekday categorization results in the following values shown in Table \ref{Weekday categorization results}.
\begin{table}[h]
\centering
\caption{Weekday categorization without slots}
\label{Weekday categorization results}
\scalebox{0.5}{%
\begin{tabular}{lllllll}
\hline
       & \multicolumn{3}{l}{yum2take}                                                 & \multicolumn{3}{l}{all}                                                    \\ \hline
Algorithm   & \begin{tabular}[c]{@{}l@{}}training\\ \#686\end{tabular} & \begin{tabular}[c]{@{}l@{}}test\\ \#945\end{tabular} & difference & \begin{tabular}[c]{@{}l@{}}training\\ \#2129\end{tabular} & \begin{tabular}[c]{@{}l@{}}test\\ \#3196\end{tabular} & difference \\ \hline
MA      & 6.208329                         & 6.07587                       & -0.13245  & 8.582432                         & 8.660576                       & 0.078144  \\ \hline
MA (8/12)   & 6.181677                         & 6.105938                       & -0.07573  & 8.08649                          & 8.700854                       & 0.614363  \\ \hline
SES (0.2/0.3) & 6.23239                         & 6.113957                       & -0.11843  & 8.573267                         & 8.683558                       & 0.110291     \\ \hline
\end{tabular}}
\end{table}
\newline\newline\textbf{No restaurant categorization}\newline
When using weekdays as categorization, the Moving Average model 8.6 minutes as RMSE. There can be made 3 models for the Weighted Moving Average using 4, 8 and 12 weeks as weight. The results improve the more weeks are taken as weight. The first weight with 4 weeks has a RMSE of 9.1 minutes, the second one improves to 8.5 minutes while the last one is the best with an RMSE of 8.1 minutes. The Simple Exponential Smoothing model using an $\alpha$ of 0.2 has the best result with 8.6 minutes. All other models with this algorithm range from a little above 8.6 minutes up to 8.9 minutes, increasing with growing $\alpha$. When compared to the test set all results decrease in accuracy. The Moving Average model decreases by 0.08 minutes while the Weighted Moving Average increases by 0.05 minutes for a weight of 4 weeks, but decreases by 0.24 minutes for 8 weeks and 0.61 minutes for 12 weeks. The Simple Exponential Smoothing models also decrease between 0.06 minutes to 0.12 minutes.\newline
This categorization results in an improvement of around 5\% over the basic approach.
\newline\newline\textbf{Restaurant categorization using yum2take}\newline
Weekday categorization returns 6.2 minutes for the Moving Average model. It improves by 0.13 minutes for the test set. The Weighted Moving Average has three different weights, 4, 8 and 12 weekdays in a row. The results are 6.7 minutes, 6.2 minutes and 6.5 minutes and they improve by 0.4 minutes for the first and thirds and by 0.07 for the second when used on the test set. When using Simple Exponential Smoothing for this categorization the results start with 6.3 minutes of RMSE for all $\alpha$ under 0.6 and grow up to 6.9 minutes when $\alpha$ is 1. The accuracy improvement from training to test set is around 0.1 minute and then grows equally to 0.36 minutes.\newline
When using this categorization an improvement of 27\% over the basic approach can be achieved by using the Moving Average.
\subsection{With slot categorization}
The results of weekday categorization and slots is shown in Table \ref{Weekday and slot categorization results}.

\begin{table}[h]
\centering
\caption{Weekday categorization with slots}
\label{Weekday and slot categorization results}
\scalebox{0.5}{%
\begin{tabular}{lllllll}
\hline
       & \multicolumn{3}{l}{yum2take}                                                 & \multicolumn{3}{l}{all}                                                    \\ \hline
Algorithm   & \begin{tabular}[c]{@{}l@{}}training\\ \#686\end{tabular} & \begin{tabular}[c]{@{}l@{}}test\\ \#945\end{tabular} & difference & \begin{tabular}[c]{@{}l@{}}training\\ \#2129\end{tabular} & \begin{tabular}[c]{@{}l@{}}test\\ \#3196\end{tabular} & difference \\ \hline
MA      & 6.063212                         & 6.13718                       & 0.073968  & 8.448001                         & 8.73583                        & 0.287828  \\ \hline
MA (12/12)  & 4.202483                         & 5.802201                       & 1.599717  & 8.117922                         & 8.913511                       & 0.795589  \\ \hline
SES (0.3/0.2) & 6.319075                         & 6.163683                       & -0.15539  & 8.378466                         & 8.720113                       &      \\ \hline
\end{tabular}}
\end{table}
\newline\newline\textbf{No restaurant categorization}\newline
The results for the Moving Average model is matching other results with around 8.4 minutes. The weights of 4 orders, 8 orders and 12 orders return results of 8.6 minutes, 8.3 minutes and 8.1 minutes. The Simple Exponential Smoothing models are in the range from 8.4 minutes up to 9.6 minutes. Regarding the accuracy when compared to the test set, the Moving Average model declines by 0.28 minutes while the 4, 8 and 12 orders weight increase by over 0.43 minutes. An increase has the $\alpha$ of 1 with 0.07 minutes while the rest decreases its accuracy with results between 0.02 minutes and 0.34 minutes.\newline
Using this categorization and the best Weighted Moving Average the results is an improvement of around 5\% over the basic approach.
\newline\newline\textbf{Restaurant categorization using yum2take}\newline
The weekday and slot categorization has an RMSE of 6.1 minutes for the Moving Average and 6.8 minutes, 6.3 minutes and an outlier of 4.4 minutes for the Weighted Moving Average with an weight of 4, 8 and 12 weeks as weight. The difference between training set and test set is 0.07 minutes of decrease for the Moving Average and 0.36 minutes of improvement for the first two weights of the Weighted Moving Average. The outlier decreases its accuracy by 1.6 minutes to 5.8 minutes which is a "normal" result. The results for the Simple Exponential Smoothing are around 6.4 minutes for $\alpha$ from 0.1 to 0.6 and growing to 6.77 minutes for bigger $\alpha$. The difference is $\pm$0.1 minute for these models.\newline
The results of this categorization, when leaving outliers out, are an increase in accuracy of 28\% over the basic approach.
\section{Mini slot categorization}\label{section:Single slot categorization}
The results of mini slot categorization are shown in Table \ref{Single slot categorization results}. There is no Weighted Moving Average since some slots do not have enough data to calculate this model.
\begin{table}[h]
\centering
\caption{Mini slot categorization results}
\label{Single slot categorization results}
\scalebox{0.5}{%
\begin{tabular}{lllllll}
\hline
       & \multicolumn{3}{l}{yum2take}                                                 & \multicolumn{3}{l}{all}                                                    \\ \hline
Algorithm   & \begin{tabular}[c]{@{}l@{}}training\\ \#686\end{tabular} & \begin{tabular}[c]{@{}l@{}}test\\ \#945\end{tabular} & difference & \begin{tabular}[c]{@{}l@{}}training\\ \#2129\end{tabular} & \begin{tabular}[c]{@{}l@{}}test\\ \#3196\end{tabular} & difference \\ \hline
MA      & 6.206459                         & 6.078068                       & -0.12839  & 8.122773                         & 8.408003                       & 0.28523  \\ \hline
SES (0.1/0.2) & 6.347103                         & 6.149942                       & -0.19716  & 8.670448                         & 8.752016                       & 0.081567     \\ \hline
\end{tabular}}
\end{table}
\newline\newline\textbf{No restaurant categorization}\newline
For the minislot categorization a RMSE of 8.1 minutes is calculated for the Moving Average model. The models using Simple Exponential Smoothing range from 8.7 minutes for an $\alpha$ of 0.2 to 11.0 minutes for the maximum $\alpha$. When compared with the Test Data Set, the training set was 0.29 minutes less accurate for the Moving Average and between 0.1 and 0.3 minutes for the simple exponential alternatives.\newline
When using this categorization an improvement of around 5\% over the basic approach can be achieved.
\newline\newline\textbf{Restaurant categorization using yum2take}\newline
The Moving Average model returns 6.2 minutes RMSE and an improvement in accuracy of 0.13 minutes when using the test set. The Simple Exponential Smoothing models start with a RMSE of 6.3 minutes for the $\alpha$ of 0.1 and increase to over 8.0 minutes for an $\alpha$ of 1. The increase in accuracy for all models with this algorithm is below 0.20 minutes.\newline
This categorization is an increase of 27\% over the basic approach when using the Moving Average algorithm.
\section{Combination of Categorizations}\label{section:Combination of Categorizations}
The combination models contain not only one time categorization, but a combination of six time categories.
The results are presented differently, since there are over thirty thousand combinations for the weights and algorithm used. Only the 5 best results for the test set are shown. In additional to the RMSE, the algorithm used and the weighting of the different time categorizations is included.
\newline\newline\textbf{No restaurant categorization}\newline
When searching for the lowest error on the training set, the result is about 8.947 minutes. It is the result of different combinations. The most occurring one is a mix from 20\% of the forecast of the current slot and 25\% of the forecast from the current day. The rest is mostly created from the last 7 day's forecast as well as the slot in the last 4 weeks. These combinations produce the most accurate forecasts. Overall there are only good forecasts generated by the normal Moving Average. When using the test set to compare the result to the training set the accuracy drops by around 0.06 minutes for the top five models.\newline
When using one of these models the forecast gets worse by around 6\% than the basic approach.
\begin{table}[h]
\centering
\caption{Combination of categorizations results for all orders}
\label{Combination of Categorizations results for all orders}
\scalebox{0.5}{%
\begin{tabular}{llllllllll}
\hline
yum2take & \multicolumn{3}{l}{Results}         & \multicolumn{6}{l}{Weights}                                      \\ \hline
Algorithm & training   & test     & difference & \multicolumn{3}{l}{slot} & \multicolumn{3}{l}{all orders}                       \\ \hline
     & \#2129 orders & \#3196 orders &       & today & 7 days & 4 weeks & today & 7 days & \begin{tabular}[c]{@{}l@{}}weekday\\ 4 weeks\end{tabular} \\ \hline
MA    & 8.947472944  & 9.009768419  & 0.062295475 & 20  & 10   & 15   & 25  & 30   & 0                             \\ \hline
MA    & 8.947526044  & 9.009875609  & 0.062349565 & 20  & 5   & 20   & 25  & 30   & 0                             \\ \hline
MA    & 8.947812087  & 9.00973973  & 0.061927643 & 20  & 15   & 10   & 25  & 30   & 0                             \\ \hline
MA    & 8.947971379  & 9.010061298  & 0.062089919 & 20  & 0   & 25   & 25  & 30   & 0                             \\ \hline
MA    & 8.948138981  & 9.009792703  & 0.061653722 & 20  & 10   & 15   & 25  & 25   & 5                             \\ \hline
\end{tabular}}
\end{table}
\newline\newline\textbf{Restaurant categorization using yum2take}\newline
The results show that the best combination for yum2take consists of 25\% current slot's forecast, 15\% of the current day's forecast and 20\% of the forecast of the slot for the predecessing 7 days. The rest of weight is combined from the other forecasts and all forecasts are using Moving Average as algorithm. This results in an error of about 6.26 minutes. The improvement of the test set over the training set is around 0.15 minutes for all of the five best results.\newline
These models improve the RMSE in contrast to the one of the basic approach by around 26\%.
\begin{table}[h]
\centering
\caption{Combination of categorizations results for yum2take}
\label{Combination of Categorizations results for yum2take}
\scalebox{0.5}{%
\begin{tabular}{llllllllll}
\hline
all orders & \multicolumn{3}{l}{Results}        & \multicolumn{6}{l}{Weights}                                      \\ \hline
Algorithm & training   & test     & difference  & \multicolumn{3}{l}{slot} & \multicolumn{3}{l}{all orders}                       \\ \hline
     & \#686 orders & \#945 orders &       & today & 7 days & 4 weeks & today & 7 days & \begin{tabular}[c]{@{}l@{}}weekday\\ 4 weeks\end{tabular} \\ \hline
MA    & 6.265421304 & 6.118386791 & -0.147034513 & 25  & 20   & 15   & 15  & 0   & 25                            \\ \hline
MA    & 6.265551527 & 6.118048866 & -0.147502661 & 25  & 20   & 15   & 15  & 5   & 20                            \\ \hline
MA    & 6.265856228 & 6.117902793 & -0.147953435 & 25  & 15   & 20   & 15  & 0   & 25                            \\ \hline
MA    & 6.265868384 & 6.117507337 & -0.148361047 & 25  & 15   & 20   & 15  & 5   & 20                            \\ \hline
MA    & 6.265946422 & 6.117772134 & -0.148174288 & 25  & 20   & 15   & 15  & 10   & 15                            \\ \hline
\end{tabular}}
\end{table}
